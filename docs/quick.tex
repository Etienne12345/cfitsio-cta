\documentclass[11pt]{article}
\input{html.sty}
\htmladdtonavigation
   {\begin{rawhtml}
 <A HREF="http://heasarc.gsfc.nasa.gov/docs/software/fitsio/fitsio.html">FITSIO Home</A>
    \end{rawhtml}}

\oddsidemargin=0.20in
\evensidemargin=0.20in
\textwidth=15.5truecm
\textheight=21.5truecm

\title{CFITSIO Quick Start Guide}
\author{William Pence \thanks{HEASARC, NASA Goddard Space Flight Center,
{\it William.D.Pence@nasa.gov}}}

\date{January 2003}

\begin{document}

\maketitle
\tableofcontents

% ===================================================================
\section{Introduction}

This document is intended to help you quickly start writing C programs
to read and write FITS files using the CFITSIO library.  It covers the
most important CFITSIO routines that are needed to perform most types
of operations on FITS files. For more complete information about these
and all the other available routines in the library please refer to
the  ``CFITSIO User's Reference Guide'', which is available from the
CFITSIO Web site at {\tt http://heasarc.gsfc.nasa.gov/fitsio}.

For more general information about the FITS data format, refer to the
following web page:
http://heasarc.gsfc.nasa.gov/docs/heasarc/fits.html

FITS stands for Flexible Image Transport System and is the standard
file format used to store most astronomical data files.  There are 2
basic types of FITS files: images and tables.  FITS images often
contain a 2-dimensional array of pixels representing an image of a
piece of the sky, but  FITS images can also contain 1-D arrays (i.e,
a spectrum or light curve), or  3-D arrays (a data cube), or
even higher dimensional arrays of data.   An image may also have zero
dimensions, in which case it is referred to as a null or empty array.
The supported datatypes for the image arrays are 8, 16, and 32-bit
integers, and 32 and 64-bit floating point real numbers.  Both signed
and unsigned integers are supported.

FITS tables contain rows and columns of data, similar to a
spreadsheet.  All the values in a particular column must have the same
datatype.  A cell of a column is not restricted to a single number, and
instead can contain an array or vector of numbers.  There are actually
2 subtypes of FITS tables: ASCII and binary. As the names imply,  ASCII
tables store the data values in an ASCII representation whereas binary
tables store the data values in a more efficient machine-readable
binary format.  Binary tables are generally more compact and support
more features (e.g., a wider range of datatypes, and vector columns)
than ASCII tables.

A single FITS file many contain multiple images or tables.  Each table
or image is called a Header-Data Unit, or HDU.  The first HDU in a FITS
file must be an image (but it may have zero axes) and is called the
Primary Array.  Any additional HDUs in the file (which are also
referred to as `extensions') may contain either an image or a table.

Every HDU contains a header containing keyword records.  Each keyword
record is 80 ASCII characters long and has the following format:

\begin{verbatim}
KEYWORD = value / comment string
\end{verbatim}

The keyword name can be up to 8 characters long (all uppercase).  The
value can be either an integer or floating point number, a logical
value (T or F), or a character string enclosed in single quotes.  Each
header begins with a series of required keywords to describe the
datatype and format of the following data unit, if any.  Any number of
other optional keywords can be included in  the header to provide other
descriptive information about the data.  For the most part, the CFITSIO
routines automatically write the required FITS keywords for each HDU,
so you, the programmer, usually do not need to worry about them.

% ===================================================================
\section{Installing and Using CFITSIO}

First, you should download the CFITSIO software and the set of example
FITS utility programs from the web site at
http://heasarc.gsfc.nasa.gov/fitsio.  The example programs illustrate
how to perform many common types of operations on FITS files using
CFITSIO.  They are also useful when writing a new program because it is
often easier to take a copy of one of these utility programs as a
template and then modify it for your own purposes, rather than writing
the new program completely from scratch.

To build the CFITSIO library on Unix platforms, `untar' the source code
distribution file and then execute the following commands in the
directory containing the source code:

\begin{verbatim}
>  ./configure [--prefix=/target/installation/path]
>  make           (or 'make shared')
>  make install   (this step is optional)
\end{verbatim}

The optional
'prefix' argument to configure gives the path to the directory where
the CFITSIO library and include files should be installed via the later
'make install' command. For example,

\begin{verbatim}
>  ./configure --prefix=/usr1/local
\end{verbatim}

will cause the 'make install' command to copy the CFITSIO libcfitsio file 
to /usr1/local/lib and the necessary include files to /usr1/local/include
(assuming of course that the  process has permission to write to these 
directories).

Pre-compiled versions of the CFITSIO DLL library are available for
PCs.  On Macintosh machines, refer to the README.MacOS file for
instructions on building CFITSIO using CodeWarrior.

Any programs that use CFITSIO must of course be linked with the CFITSIO
library when creating the executable file.  The exact procedure for
linking a program depends on your software environment, but on Unix
platforms, the command line to compile and link a program will look
something like this:

\begin{verbatim}
gcc -o myprog myprog.c -L. -lcfitsio -lm -lnsl -lsocket
\end{verbatim}

You may not need to include all of the 'm', 'nsl', and 'socket' system
libraries on your particular machine.  To find out what libraries are
required on your (Unix) system, type {\tt'make testprog'} and see what
libraries are then included on the resulting link line.

\newpage
% ===================================================================
\section{Example Programs}

Before describing the individual CFITSIO routines in detail, it is
instructive to first look at an actual program.  The names of the
CFITSIO routines are fairly descriptive (they all begin with {\tt
fits\_}, so it should be reasonably clear what this program does:

\begin{verbatim}
----------------------------------------------------------------
    #include <string.h>
    #include <stdio.h>
1:  #include "fitsio.h"

    int main(int argc, char *argv[])
    {
2:      fitsfile *fptr;         
        char card[FLEN_CARD]; 
3:      int status = 0,  nkeys, ii;  /* MUST initialize status */

4:      fits_open_file(&fptr, argv[1], READONLY, &status);
        fits_get_hdrspace(fptr, &nkeys, NULL, &status);

        for (ii = 1; ii <= nkeys; ii++)  { 
          fits_read_record(fptr, ii, card, &status); /* read keyword */
          printf("%s\n", card);
        }
        printf("END\n\n");  /* terminate listing with END */
        fits_close_file(fptr, &status);

        if (status)          /* print any error messages */
5:          fits_report_error(stderr, status);
        return(status);
    }
----------------------------------------------------------------
\end{verbatim}

This program opens the specified FITS file and prints
out all the header keywords in the current HDU.
Some other points to notice about the program are:
\begin{enumerate}

\item
The {\tt fitsio.h} header file must be included to define the 
various routines and symbols used in CFITSIO.

\item

The {\tt fitsfile}  parameter is the first argument in almost every
CFITSIO routine.  It is a pointer to a structure (defined in {\tt
fitsio.h}) that stores information about the particular FITS file that
the routine will operate on.  Memory for this structure is
automatically allocated when the file is first opened or created, and
is freed when the file is closed.

\item
Almost every CFITSIO routine has a {\tt status} parameter as the last
argument. The status value is also usually returned as the value of the
function itself.  Normally status = 0, and a positive status value
indicates an error of some sort.  The status variable must always be
initialized to zero before use, because if status is greater than zero
on input then the CFITSIO routines will simply return without doing
anything.  This `inherited status' feature, where each CFITSIO routine
inherits the status from the previous routine, makes it unnecessary to
check the status value after every single CFITSIO routine call.
Generally you should check the status after an especially important or
complicated routine has been called, or after a block of
closely related CFITSIO calls.  This example program has taken this
feature to the extreme and only checks the status value at the 
very end of the program.

\item

In this example program the file name to be opened is given as an
argument on the command line ({\tt arg[1]}).  If the file contains more
than 1 HDU or extension, you can specify which particular HDU to be
opened by enclosing the name or number of the HDU in square brackets
following the root name of the file.  For example, {\tt file.fts[0]}
opens the  primary array, while {\tt file.fts[2]} will move to and open
the 2nd extension in the file, and {\tt file.fit[EVENTS]} will open the
extension that has a {\tt EXTNAME = 'EVENTS'} keyword in the header.
Note that on the Unix command line you must enclose the file name in
single or double quote characters if the name contains special
characters such as `[' or `]'.

All of the CFITSIO routines which read or write header keywords,
image data, or table data operate only within the currently opened
HDU in the file. To read or write information in a different HDU you must
first explicitly move to that HDU (see the {\tt fits\_movabs\_hdu} and
{\tt fits\_movrel\_hdu} routines in section 4.3).

\item

The {\tt fits\_report\_error} routine provides a convenient way to print out
diagnostic messages about any error that may have occurred. 

\end{enumerate}

A set of example FITS utility programs are  available from the CFITSIO
web site at \newline
http://heasarc.gsfc.nasa.gov/docs/software/fitsio/cexamples.html.
These are real working programs which illustrate how to read, write,
and modify FITS files using the CFITSIO library.  Most of these
programs are very short, containing only a few 10s of lines of
executable code or less, yet they perform quite useful operations on
FITS files. Running each program without any command line arguments
will produce a short description of how to use the program.
The currently available programs are:
\begin{quote}
fitscopy - copy a file
\newline
listhead - list header keywords
\newline
liststruc - show the structure of a FITS file.
\newline
modhead  - write or modify a header keyword
\newline
imarith  - add, subtract, multiply, or divide 2 images
\newline
imlist  - list pixel values in an image
\newline
imstat  - compute mean, min, and max pixel values in an image
\newline
tablist - display the contents of a FITS table
\newline
tabcalc  - general table calculator
\end{quote}

\newpage

% ===================================================================
\section{CFITSIO Routines}

This chapter describes the main CFITSIO routines that can be used to
perform the most common types of operations on FITS files.

% ===================================================================
{\bf \subsection{Error Reporting}}

\begin{verbatim}
void fits_report_error(FILE *stream, int status)
void fits_get_errstatus(int status, char *err_text)
float fits_get_version(float *version)
\end{verbatim}

The first routine prints out information about any error that
has occurred.  Whenever any CFITSIO routine encounters an error it
usually writes a message describing the nature of the error to an
internal error message stack and then returns with a positive integer
status value. Passing the error status value to this routine will
cause  a generic description of the error and all the messages
from the internal CFITSIO error stack to be printed to the specified
stream.  The {\tt stream} parameter is usually set equal to
{\tt "stdout"} or {\tt "stderr"}.

The second routine simply returns a 30-character descriptive
error message corresponding to the input status value.

The last routine returns the current CFITSIO library version number.

% ===================================================================
{\bf \subsection{File Open/Close Routines}}

\begin{verbatim}
int fits_open_file( fitsfile **fptr, char *filename, int mode, int *status)
int fits_open_data( fitsfile **fptr, char *filename, int mode, int *status)
int fits_open_table(fitsfile **fptr, char *filename, int mode, int *status)
int fits_open_image(fitsfile **fptr, char *filename, int mode, int *status)

int fits_create_file(fitsfile **fptr, char *filename, int *status)
int fits_close_file(fitsfile *fptr, int *status)
\end{verbatim}

These routines open or close a file.  The first {\tt fitsfile}
parameter  in these and nearly every other CFITSIO routine is a pointer
to a structure that CFITSIO uses to store relevant parameters about
each opened file.  You should never directly read or write any
information in this structure.  Memory for this structure is allocated
automatically when the file is opened or created, and is freed when the
file is closed.

The {\tt mode} parameter in the {\tt fits\_open\_xxxx} set of routines
can be set to either {\tt READONLY} or {\tt READWRITE} to select the
type of file access that will be allowed. These symbolic constants are
defined in {\tt fitsio.h}.

The {\tt fits\_open\_file} routine opens the file and positions the internal
file pointer to the beginning of the file, or to the specified
extension if an extension name or number is appended to the file name
(see the later section on ``CFITSIO File Names and Filters'' for a
description of the syntax). {\tt fits\_open\_data} behaves similarly except
that it will move to the first HDU containing significant data if a HDU
name or number to open is not explicitly specified as part of the
filename.  It will move to the first IMAGE HDU with NAXIS greater than
0, or the first table that does not contain the strings `GTI' (a Good
Time Interval extension) or `OBSTABLE' in the EXTNAME keyword value.
The {\tt fits\_open\_table} and {\tt fits\_open\_image}  routines are similar
except that they will move to the first significant table HDU or image
HDU, respectively if a HDU name of number is not specified as part of
the input file name.

When opening an existing file, the {\tt filename} can include optional
arguments, enclosed in square brackets that specify filtering
operations that should be applied to the input file.  For example,
\begin{verbatim}
   myfile.fit[EVENTS][counts > 0]
\end{verbatim}
opens the table in the EVENTS extension and creates a virtual table by
selecting only those rows where the COUNTS column value is greater than
0.  See section 5 for more examples of these powerful filtering
capabilities.

In {\tt fits\_create\_file},  the {\tt filename} is simply the root name of
the file to be created.  You can overwrite an existing file by
prefixing the name with a `!' character (on the Unix command line this
must be prefixed with a backslash, as in \verb+`\!file.fit'+).  
If the file name ends with {\tt .gz} the file will be compressed
using the gzip algorithm.  If the
filename is {\tt stdout} or {\tt "-"} (a single dash character)
then the output file will be piped to the stdout stream.  You can
chain several tasks together by writing the output from the first task
to {\tt stdout} and then reading the input file in the 2nd task from
{\tt stdin} or {\tt "-"}.


% ===================================================================
{\bf \subsection{HDU-level Routines}}

The routines listed in this section operate on Header-Data Units (HDUs) in a file.

\begin{verbatim}
_______________________________________________________________
int fits_get_num_hdus(fitsfile *fptr, int *hdunum, int *status)
int fits_get_hdu_num(fitsfile *fptr,  int *hdunum)
\end{verbatim}

The first routines returns the total number of HDUs in the FITS file,
and the second routine returns the position of the currently opened HDU in
the FITS file (starting with 1, not 0).

\begin{verbatim}
__________________________________________________________________________
int fits_movabs_hdu(fitsfile *fptr, int hdunum, int *hdutype, int *status)
int fits_movrel_hdu(fitsfile *fptr, int nmove,  int *hdutype, int *status)
int fits_movnam_hdu(fitsfile *fptr, int hdutype, char *extname,
                    int extver, int *status)
\end{verbatim}

These routines enable you to move to a different HDU in the file.
Most of the CFITSIO functions which read or write keywords or data
operate only on the currently opened HDU in the file.  The first
routine moves to the specified absolute HDU number in the FITS
file (the first HDU = 1), whereas the second routine moves a relative
number of HDUs forward or backward from the currently open HDU.  The
{\tt hdutype} parameter returns the type of the newly opened HDU, and will
be equal to one of these symbolic constant values: {\tt IMAGE\_HDU,
ASCII\_TBL, or BINARY\_TBL}.  {\tt hdutype} may be set to NULL
if it is not needed.  The third routine moves to the (first) HDU
that matches the input extension type, name, and version number,
as given by the {\tt XTENSION, EXTNAME} (or {\tt HDUNAME}) and {\tt EXTVER} keywords.
If the input value of {\tt extver} = 0, then the version number will
be ignored when looking for a matching HDU.

\begin{verbatim}
_________________________________________________________________
int fits_get_hdu_type(fitsfile *fptr,  int *hdutype, int *status)
\end{verbatim}

Get the type of the current HDU in the FITS file:  {\tt IMAGE\_HDU,
ASCII\_TBL, or BINARY\_TBL}.

\begin{verbatim}
____________________________________________________________________
int fits_copy_hdu(fitsfile *infptr, fitsfile *outfptr, int morekeys,
                  int *status)
int fits_copy_file(fitsfile *infptr, fitsfile *outfptr, int previous,
                  int current, int following, > int *status)
\end{verbatim}

The first routine copies the current HDU from the FITS file associated
with infptr and appends it to the end of the FITS file associated with
outfptr.  Space may be reserved for {\tt morekeys} additional keywords
in the output header.   The second routine copies any HDUs previous
to the current HDU, and/or the current HDU, and/or any HDUs following the
current HDU, depending on the value (True or False) of {\tt previous,
current}, and {\tt following}, respectively.  For example,
\begin{verbatim}
  fits_copy_file(infptr, outfptr, 0, 1, 1, &status);
\end{verbatim}
will copy the current HDU and any HDUs that follow it from the input
to the output file, but it will not copy any HDUs preceding the 
current HDU.


\newpage
% ===================================================================
\subsection{Image I/O Routines}

This section lists the more important CFITSIO routines which operate on
FITS images.

\begin{verbatim}
_______________________________________________________________
int fits_get_img_type(fitsfile *fptr, int *bitpix, int *status)
int fits_get_img_dim( fitsfile *fptr, int *naxis,  int *status)
int fits_get_img_size(fitsfile *fptr, int maxdim,  long *naxes,
                      int *status)
int fits_get_img_param(fitsfile *fptr, int maxdim,  int *bitpix,
                       int *naxis, long *naxes, int *status)
\end{verbatim}

Get information about the currently opened image HDU. The first routine
returns the datatype of the image as (defined by the {\tt BITPIX}
keyword), which can have the following symbolic constant values: 
\begin{verbatim}
    BYTE_IMG   =   8   ( 8-bit byte pixels, 0 - 255)
    SHORT_IMG  =  16   (16 bit integer pixels)
    LONG_IMG   =  32   (32-bit integer pixels)
    FLOAT_IMG  = -32   (32-bit floating point pixels)
    DOUBLE_IMG = -64   (64-bit floating point pixels)
\end{verbatim}

The second and third routines return the number of dimensions in the
image (from the {\tt NAXIS} keyword), and the sizes of each dimension
(from the {\tt NAXIS1, NAXIS2}, etc. keywords).  The last routine
simply combines the function of the first 3 routines.  The input {\tt
maxdim} parameter in this routine gives the maximum number dimensions
that may be returned (i.e., the dimension of the {\tt naxes}
array)

\begin{verbatim}
__________________________________________________________
int fits_create_img(fitsfile *fptr, int bitpix, int naxis, 
                    long *naxes, int *status)
\end{verbatim}

Create an image HDU by writing the required keywords which define the
structure of the image.  The 2nd through 4th parameters  specified the
datatype, the number of dimensions, and the sizes of the dimensions.
The allowed values of the {\tt bitpix} parameter are listed above in
the description of the {\tt fits\_get\_img\_type} routine.  If the FITS
file pointed to by {\tt fptr} is empty (previously created with
{\tt fits\_create\_file}) then this routine creates a primary array in
the file, otherwise a new IMAGE extension is appended to end of the
file following the other HDUs in the file.

\begin{verbatim}
______________________________________________________________
int fits_write_pix(fitsfile *fptr, int datatype, long *fpixel,
               long nelements, void *array, int *status);

int fits_write_pixnull(fitsfile *fptr, int datatype, long *fpixel,
               long nelements, void *array, void *nulval, int *status);

int fits_read_pix(fitsfile *fptr, int  datatype, long *fpixel, 
                  long nelements, void *nulval, void *array, 
                  int *anynul, int *status)
\end{verbatim}

Read or write all or part of the FITS image.  There are 2 different
'write' pixel routines:  The first simply writes the input array of pixels
to the FITS file.  The second is similar, except that it substitutes
the appropriate null pixel value in the FITS file for any pixels 
which have a value equal to {\tt *nulval} (note that this parameter
gives the address of the null pixel value, not the value itself).
Similarly,  when reading an image, CFITSIO will substitute the value
given by {\tt nulval}  for  any undefined pixels in the image, unless
{\tt nulval = NULL}, in which case no checks will be made for undefined
pixels when reading the FITS image.

The {\tt fpixel} parameter in these routines is an array which gives
the coordinate in each dimension of the first pixel to be read or
written, and {\tt nelements} is the total number of pixels to read or
write.  {\tt array} is the address of an array which either contains
the pixel values to be written, or will hold the values of the pixels
that are read.  When reading, {\tt array} must have been allocated
large enough to hold all the returned pixel values.  These routines
starts at the {\tt fpixel} location and then read or write the {\tt
nelements} pixels, continuing on successive rows of the image if
necessary.  For example, to write an entire 2D image, set {\tt
fpixel[0] = fpixel[1] = 1}, and {\tt nelements = NAXIS1 * NAXIS2}.  Or
to read just the 10th row of the image, set {\tt fpixel[0] = 1,
fpixel[1] = 10}, and {\tt nelements = NAXIS1}.  The {\tt datatype}
parameter specifies the datatype of the C {\tt array} in the program,
which need not be the same as the datatype of the FITS image itself.
If the datatypes differ then CFITSIO will convert the data as it is
read or written.  The following symbolic constants are allowed for the
value of {\tt datatype}:
\begin{verbatim}
  TBYTE     unsigned char
  TSBYTE    signed char
  TSHORT    signed short
  TUSHORT   unsigned short
  TINT      signed int
  TUINT     unsigned int
  TLONG     signed long
  TLONGLONG signed 8-byte integer
  TULONG    unsigned long
  TFLOAT    float
  TDOUBLE   double
\end{verbatim}


\begin{verbatim}
_________________________________________________________________
int fits_write_subset(fitsfile *fptr, int datatype, long *fpixel,
             long *lpixel, DTYPE *array, > int *status)

int fits_read_subset(fitsfile *fptr, int  datatype, long *fpixel,
             long *lpixel, long *inc, void *nulval,  void *array,
             int *anynul, int *status)
\end{verbatim}

Read or write a rectangular section of the FITS image.  These are very
similar to {\tt fits\_write\_pix} and {\tt fits\_read\_pix} except that
you specify the last pixel coordinate (the upper right corner of the
section) instead of the number of pixels to be read.  The read routine
also has an {\tt inc} parameter which can be used to read only every
{\tt inc-th} pixel along each dimension of the image.  Normally  {\tt
inc[0] = inc[1] = 1} to read every pixel in a 2D image.  To read every
other pixel in the entire 2D image, set
\begin{verbatim}
    fpixel[0] = fpixel[1] = 1
    lpixel[0] = {NAXIS1}
    lpixel[1] = {NAXIS2}  
    inc[0] = inc[1] = 2  
\end{verbatim}

Or, to read the 8th row of a 2D image, set 
\begin{verbatim}
    fpixel[0] = 1
    fpixel[1] = 8
    lpixel[0] = {NAXIS1}
    lpixel[1] = 8
    inc[0] = inc[1] = 1
\end{verbatim}

\newpage
% ===================================================================
\subsection{Table I/O Routines}

This section lists the most important CFITSIO routines which operate on
FITS tables.

\begin{verbatim}
__________________________________________________________________________
int fits_create_tbl(fitsfile *fptr, int tbltype, long nrows, int tfields,
    char *ttype[],char *tform[], char *tunit[], char *extname, int *status)
\end{verbatim}

Create a new  table extension by writing the required keywords that
define the table structure. The required null primary array
will be created first if the file is initially completely empty.  {\tt
tbltype} defines the type of table and can have values of {\tt
ASCII\_TBL or BINARY\_TBL}.  Binary tables are generally preferred
because they are more efficient and support a greater range of column
datatypes than ASCII tables.

The {\tt nrows} parameter gives the initial number of empty rows to be
allocated for the table; this should normally be set to 0.  The {\tt tfields}
parameter gives the number of columns in the table (maximum = 999).
The {\tt
ttype, tform}, and {\tt tunit} parameters give the name, datatype, and
physical units of each column, and {\tt extname} gives the name for the
table (the value of the {\tt EXTNAME} keyword).  
The FITS Standard recommends that only
letters, digits, and the underscore character be used in column names
with no embedded spaces.  It is recommended that all the column names
in a given table be unique within the first 8 characters.

The following table
shows the TFORM column format values that are allowed in ASCII tables
and in binary tables:
\begin{verbatim}
        ASCII Table Column Format Codes
        -------------------------------
        (w = column width, d = no. of decimal places to display)
            Aw   - character string
            Iw   - integer
            Fw.d - fixed floating point
            Ew.d - exponential floating point
            Dw.d - exponential floating point

        Binary Table Column Format Codes
        --------------------------------
        (r = vector length, default = 1)
            rA  - character string
            rAw - array of strings, each of length w
            rL  - logical
            rX  - bit
            rB  - unsigned byte
            rS  - signed byte **
            rI  - signed 16-bit integer
            rU  - unsigned 16-bit integer **
            rJ  - signed 32-bit integer
            rV  - unsigned 32-bit integer **
            rK  - 64-bit integer ***
            rE  - 32-bit floating point
            rD  - 64-bit floating point
            rC  - 32-bit complex pair
            rM  - 64-bit complex pair

     ** The S, U and V format codes are not actual legal TFORMn values.
        CFITSIO substitutes the somewhat more complicated set of
        keywords that are used to represent unsigned integers or
        signed bytes.

    *** The 64-bit integer format is experimental and is not 
        officially recognized in the FITS Standard.
\end{verbatim}
  
The {\tt tunit} and {\tt extname} parameters are optional and
may be set to NULL
if they are not needed.  

Note that it may be easier to create a new table by copying the
header from another existing table with {\tt fits\_copy\_header} rather
than calling this routine.

\begin{verbatim}
_______________________________________________________________
int fits_get_num_rows(fitsfile *fptr, long *nrows, int *status)
int fits_get_num_cols(fitsfile *fptr, int  *ncols, int *status)
\end{verbatim}

Get the number of rows or columns in the current FITS table.  The
number of rows is given by the {\tt NAXIS2} keyword and the number of columns
is given by the {\tt TFIELDS} keyword in the header of the table.

\begin{verbatim}
_______________________________________________________________
int fits_get_colnum(fitsfile *fptr, int casesen, char *template,
                    int *colnum, int *status)
int fits_get_colname(fitsfile *fptr, int casesen, char *template,
                    char *colname, int *colnum, int *status)
\end{verbatim}

Get the  column number (starting with 1, not 0) of the column whose
name matches the specified template name.  The only difference in
these 2 routines is that the 2nd one also returns the name of the
column that matched the template string.

Normally, {\tt casesen} should
be set to {\tt CASEINSEN}, but it may be set to {\tt CASESEN} to force
the name matching to be case-sensitive.

The input {\tt template} string gives the name of the desired column and
may include wildcard characters:  a `*' matches any sequence of
characters (including zero characters), `?' matches any single
character, and `\#' matches any consecutive string of decimal digits
(0-9).  If more than one column name in the table matches the template
string, then the first match is returned and the status value will be
set to {\tt COL\_NOT\_UNIQUE}  as a warning that a unique match was not
found.  To find the next column that matches the template, call this
routine again leaving the input status value equal to {\tt
COL\_NOT\_UNIQUE}.  Repeat this process until {\tt status =
COL\_NOT\_FOUND}  is returned.

\begin{verbatim}
_______________________________________________________________
int fits_get_coltype(fitsfile *fptr, int colnum, int *typecode,
                     long *repeat, long *width, int *status)

int fits_get_eqcoltype(fitsfile *fptr, int colnum, int *typecode,
                     long *repeat, long *width, int *status)
\end{verbatim}

Return the datatype, vector repeat count, and the width in bytes of a
single column element for column number {\tt colnum}.  Allowed values
for the returned datatype in ASCII tables are:  {\tt TSTRING, TSHORT,
TLONG, TFLOAT, and TDOUBLE}.  Binary tables support these additional
types: {\tt TLOGICAL, TBIT, TBYTE, TINT32BIT, TCOMPLEX and TDBLCOMPLEX}.  The
negative of the datatype code value is returned if it is a variable
length array column.

These 2 routines are similar, except that in the case of scaled
integer columns the 2nd routine, fit\_get\_eqcoltype, returns the
'equivalent' datatype that is needed to store the scaled values, which
is not necessarily the same as the physical datatype of the unscaled values
as stored in the FITS table.  For example if a '1I' column in a binary
table has TSCALn = 1 and TZEROn = 32768, then this column effectively
contains unsigned short integer values, and thus the returned value of
typecode will be TUSHORT, not TSHORT.  Or, if TSCALn or TZEROn are not
integers, then the equivalent datatype will be returned as TFLOAT or
TDOUBLE, depending on the size of the integer.

The repeat count is always 1 in ASCII tables.
The 'repeat' parameter returns the vector repeat count on the binary
table TFORMn keyword value. (ASCII table columns always have repeat
= 1).  The 'width' parameter returns the width in bytes of a single
column element (e.g., a '10D' binary table column will have width =
8, an ASCII table 'F12.2' column will have width = 12, and a binary
table'60A' character string  column will have width = 60);  Note that
this routine supports the local convention for specifying arrays of
fixed length strings within a binary table character column using
the syntax TFORM = 'rAw' where 'r' is the total number of
characters (= the width of the column) and 'w' is the width of a
unit string within the column.  Thus if the column has TFORM =
'60A12' then this means that each row of the table contains
5 12-character substrings within the 60-character field, and thus 
in this case this routine will return typecode = TSTRING, repeat =
60, and width = 12.  The number of substings in any binary table
character string field can be calculated by (repeat/width). 
A null pointer may be given for any of the output parameters that
 are not needed.

\begin{verbatim}
____________________________________________________________________________
int fits_insert_rows(fitsfile *fptr, long firstrow, long nrows, int *status)
int fits_delete_rows(fitsfile *fptr, long firstrow, long nrows, int *status)
int fits_delete_rowrange(fitsfile *fptr, char *rangelist, int *status)
int fits_delete_rowlist(fitsfile *fptr, long *rowlist, long nrows, int *stat)
\end{verbatim}

Insert or delete rows in a table.  The blank rows are inserted
immediately following row {\tt frow}. Set {\tt frow} = 0 to insert rows
at the beginning of the table.  The first 'delete' routine deletes {\tt
nrows} rows beginning with row {\tt firstrow}.   The 2nd delete routine
takes an input string listing the rows or row ranges to be deleted
(e.g., '2,4-7, 9-12').  The last delete routine takes an input long
integer array that specifies each individual row to be deleted.  The
row lists must be sorted in ascending order.  All these routines update
the value of the {\tt NAXIS2} keyword to reflect the new number of rows
in the table.

\begin{verbatim}
_________________________________________________________________________
int fits_insert_col(fitsfile *fptr, int colnum, char *ttype, char *tform,
                    int *status)
int fits_insert_cols(fitsfile *fptr, int colnum, int ncols, char **ttype,
                     char **tform, int *status)

int fits_delete_col(fitsfile *fptr, int colnum, int *status)
\end{verbatim}

Insert or delete columns in a table.  {\tt colnum} gives the position
of the column to be inserted or deleted (where the first column of the
table is at position 1).  {\tt ttype} and {\tt tform} give the column
name and column format, where the allowed format codes are listed above
in the description of the {\tt fits\_create\_table} routine.  The 2nd
'insert' routine inserts multiple columns, where {\tt ncols} is the
number of columns to insert, and  {\tt ttype} and {\tt tform} are
arrays of string pointers in this case.

\begin{verbatim}
____________________________________________________________________
int fits_copy_col(fitsfile *infptr, fitsfile *outfptr, int incolnum,
        int outcolnum, int create_col, int *status);
\end{verbatim}

Copy a column from one table HDU to another.  If {\tt create\_col} = TRUE (i.e., not equal to zero),
then a new column will be inserted in the output table at position
{\tt outcolumn}, otherwise the values in the existing output column will be
overwritten. 

\begin{verbatim}
__________________________________________________________________________
int fits_write_col(fitsfile *fptr, int datatype, int colnum, long firstrow,
                  long firstelem, long nelements, void *array, int *status)
int fits_write_colnull(fitsfile *fptr, int datatype, int colnum, 
                  long firstrow, long firstelem, long nelements, 
                  void *array, void *nulval, int *status)
int fits_write_col_null(fitsfile *fptr, int colnum, long firstrow,
                  long firstelem, long nelements, int *status)

int fits_read_col(fitsfile *fptr, int datatype, int colnum, long firstrow,
       long firstelem, long nelements, void *nulval, void *array, 
       int *anynul, int *status)

\end{verbatim}

Write or read elements in column number {\tt colnum}, starting with row
{\tt firstsrow} and element {\tt firstelem} (if it is a vector
column).  {\tt firstelem} is ignored if it is a scalar column. The {\tt
nelements} number of elements are read or written continuing on
successive rows of the table if necessary. {\tt array} is the address
of an array which either contains the  values to be written, or will
hold the returned values that are read.  When reading, {\tt array} must
have been allocated large enough to hold all the returned values.

There are 3 different 'write' column routines:  The first simply writes
the input array into the column.  The second is similar, except that it
substitutes the appropriate null pixel value in the column for any
input array values which are equal to {\tt *nulval} (note that this
parameter gives the address of the null pixel value, not the value
itself).  The third write routine sets the specified table elements
to a null value.  New rows will be automatical added to the table
if the write operation extends beyond the current size of the table.

When reading a column, CFITSIO will substitute the value given by {\tt
nulval}  for  any undefined elements in the FITS column, unless {\tt
nulval} or {\tt *nulval = NULL}, in which case no checks will be made
for undefined values when reading the column.

{\tt datatype} specifies the datatype of the C {\tt array} in the program,
which need not be the same as the intrinsic datatype of the column in
the FITS table.   The following symbolic constants are allowed for the
value of {\tt datatype}:

\begin{verbatim}
  TSTRING   array of character string pointers
  TBYTE     unsigned char
  TSHORT    signed short
  TUSHORT   unsigned short
  TINT      signed int
  TUINT     unsigned int
  TLONG     signed long
  TLONGLONG signed 8-byte integer
  TULONG    unsigned long
  TFLOAT    float
  TDOUBLE   double
\end{verbatim}

Note that {\tt TSTRING} corresponds to the C {\tt
char**} datatype, i.e., a pointer to an array of pointers to an array
of characters.

Any column, regardless of it's intrinsic datatype, may be read as a
{\tt TSTRING} character string. The display format of the returned
strings will be determined by the {\tt TDISPn} keyword, if it exists,
otherwise a default format will be used depending on the datatype of
the column.  The {\tt tablist} example utility program (available from
the CFITSIO web site) uses this feature to display all the values in a
FITS table.

\begin{verbatim}
_____________________________________________________________________
int fits_select_rows(fitsfile *infptr, fitsfile *outfptr, char *expr,
                     int *status)
int fits_calculator(fitsfile *infptr, char *expr, fitsfile *outfptr,
                    char *colname, char *tform, int *status) 
\end{verbatim}

These are 2 of the most powerful routines in the CFITSIO library.  (See
the full CFITSIO Reference Guide for a description of several related
routines).  These routines can perform complicated transformations on
tables based on an input arithmetic expression which is evaluated for
each row of the table.  The first routine will select or copy rows of
the table for which the expression evaluates to TRUE (i.e., not equal
to zero).  The second routine writes the value of the expression to a
column in the output table.  Rather than supplying the expression
directly to these routines, the expression may also be written to a
text file (continued over multiple lines if necessary) and the name of
the file, prepended with a '@' character, may be supplied as the value
of the 'expr' parameter (e.g.  '@filename.txt').

The arithmetic expression may be a function of any column or keyword in
the input table as shown in these examples:

\begin{verbatim}
Row Selection Expressions:
   counts > 0                          uses COUNTS column value
   sqrt( X**2 + Y**2) < 10.            uses X and Y column values
   (X > 10) || (X < -10) && (Y == 0)   used 'or' and 'and' operators  
   gtifilter()                         filter on Good Time Intervals
   regfilter("myregion.reg")           filter using a region file
   @select.txt                         reads expression from a text file
Calculator Expressions:
   #row % 10                        modulus of the row number
   counts/#exposure                 Fn of COUNTS column and EXPOSURE keyword
   dec < 85 ? cos(dec * #deg) : 0   Conditional expression: evaluates to
                                      cos(dec) if dec < 85, else 0
   (count{-1}+count+count{+1})/3.   running mean of the count values in the
                                      previous, current, and next rows
   max(0, min(X, 1000))             returns a value between 0 - 1000
   @calc.txt                        reads expression from a text file
\end{verbatim}

Most standard mathematical operators and functions are supported.  If
the expression includes the name of a column, than the value in the
current row of the table will be used when evaluating the expression on
each row.   An offset to an adjacent row can be specified by including
the offset value in curly brackets after the column name as shown in
one of the examples.  Keyword values can be included in the expression
by preceding the keyword name with a `\#' sign.   See Section 5 of this
document for more discussion of the expression syntax.

{\tt gtifilter} is a special function which tests whether the {\tt
TIME} column value in the input table falls within one or more Good
Time Intervals.  By default, this function looks for a 'GTI' extension
in the same file as the input table.  The 'GTI' table contains {\tt START} 
and {\tt STOP} columns which define the range of
each good time interval. See section 5.4.3 for more details.

{\tt regfilter} is another special function which selects rows based on
whether the spatial position associated with each row is located within
in a specified region of the sky.  By default, the {\tt X} and {\tt Y}
columns in the input table are assumed to give the position of each row.
The spatial region is defined in an ASCII text file whose name is given
as the argument to the {\tt regfilter} function. See section 5.4.4 for
more details.

The {\tt infptr} and {\tt outfptr} parameters in these routines may
point to the same table or to different tables.  In {\tt
fits\_select\_rows}, if the input and output tables are the same then
the rows that do not satisfy the selection expression will be deleted
from the table.  Otherwise, if the output table is different from the
input table then the selected rows will be copied from the input table
to the output table.

The output column in {\tt fits\_calculator} may or may not already
exist.  If it exists then the calculated values will be written to that
column, overwriting the existing values.  If the column doesn't exist
then the new column will be appended to the output table. The {\tt tform}
parameter can be used to specify the datatype of the new column (e.g.,
the {\tt TFORM} keyword value as in {\tt '1E', or '1J'}). If {\tt
tform} = NULL then a default datatype will be used, depending on the
expression.

\begin{verbatim}
_____________________________________________________________________
int fits_read_tblbytes(fitsfile *fptr, long firstrow, long firstchar,
                     long nchars, unsigned char *array, int *status)
int fits_write_tblbytes (fitsfile *fptr, long firstrow, long firstchar,
                     long nchars, unsigned char *array, int *status)
\end{verbatim}

These 2 routines provide low-level access to tables and are mainly
useful as an efficient way to copy rows of a table from one file to
another.  These routines simply read or write the specified number of
consecutive characters (bytes) in a table, without regard for column
boundaries.  For example, to read or write the first row of a table,
set {\tt firstrow = 1, firstchar = 1}, and {\tt nchars = NAXIS1} where
the length of a row is given by the value of the {\tt NAXIS1} header
keyword.  When reading a table, {\tt array} must have been declared at
least {\tt nchars} bytes long to hold the returned string of bytes.

\newpage
% ===================================================================
\subsection{Header Keyword I/O Routines}
\nopagebreak
The following routines read and write header keywords in the current HDU.
\nopagebreak

\begin{verbatim}
____________________________________________________________________
int fits_get_hdrspace(fitsfile *fptr, int *keysexist, int *morekeys,
                      int *status)
\end{verbatim}
\nopagebreak
Return the number of existing keywords (not counting the mandatory END
keyword) and the amount of empty space currently available for more
keywords. The {\tt morekeys} parameter may be set to NULL if it's value is
not needed.

\begin{verbatim}
___________________________________________________________________________
int fits_read_record(fitsfile *fptr, int keynum, char *record, int *status)
int fits_read_card(fitsfile *fptr, char *keyname, char *record, int *status)
int fits_read_key(fitsfile *fptr, int datatype, char *keyname,
                  void *value, char *comment, int *status)

int fits_find_nextkey(fitsfile *fptr, char **inclist, int ninc,
                      char **exclist, int nexc, char *card, int *status)

int fits_read_key_unit(fitsfile *fptr, char *keyname, char *unit, 
                       int *status)
\end{verbatim}

These routines all read a header record in the current HDU. The first
routine reads keyword number {\tt keynum} (where the first keyword is
at position 1).  This routine is most commonly used when sequentially
reading every record in the header from beginning to end.  The 2nd and
3rd routines read the named keyword and return either the whole
record, or the keyword value and comment string.  In each case any 
non-significant trailing blank characters in the strings are truncated.

Wild card characters (*, ?, and \#) may be used when specifying the name
of the keyword to be read, in which case the first matching keyword is
returned.

The {\tt datatype} parameter specifies the C datatype of the returned
keyword value and can have one of the following symbolic constant
values:  {\tt TSTRING, TLOGICAL} (== int), {\tt TBYTE}, {\tt TSHORT},
{\tt TUSHORT}, {\tt TINT}, {\tt TUINT}, {\tt TLONG}, {\tt TULONG}, {\tt
TFLOAT}, {\tt TDOUBLE}, {\tt TCOMPLEX}, and {\tt TDBLCOMPLEX}.  Data
type conversion will be performed for numeric values if the intrinsic
FITS keyword value does not have the same datatype.  The {\tt comment}
parameter may be set equal to NULL if the comment string is not
needed.

The 4th routine provides an easy way to find all the keywords in the
header that match one of the name templates in {\tt inclist} and do not
match any of the name templates in {\tt exclist}.  {\tt ninc} and {\tt
nexc} are the number of template strings in {\tt inclist} and {\tt
exclist}, respectively.  Wild cards (*, ?, and \#) may be used in the
templates to match multiple keywords.  Each time this routine is called
it returns the next matching 80-byte keyword record.  It returns status
= {\tt KEY\_NO\_EXIST} if there are no more matches.

The 5th routine returns the keyword value units string, if any.
The units are recorded at the beginning of the keyword comment field
enclosed in square brackets.
\begin{verbatim}
_______________________________________________________________
int fits_write_key(fitsfile *fptr, int datatype, char *keyname, 
        void *value, char *comment, int *status)
int fits_update_key(fitsfile *fptr, int datatype, char *keyname,
        void *value, char *comment, int *status)
int fits_write_record(fitsfile *fptr, char *card, int *status)

int fits_modify_comment(fitsfile *fptr, char *keyname, char *comment,
        int *status)
int fits_write_key_unit(fitsfile *fptr, char *keyname, char *unit,
        int *status)

\end{verbatim}

Write or modify a keyword  in the header of the current HDU.  The
first routine appends the new keyword to the end of the header, whereas
the second routine will update the value and comment fields of the
keyword if it already exists, otherwise it behaves like the first
routine and appends the new keyword.  Note that {\tt value} gives the
address to the value and not the value itself.  The {\tt datatype}
parameter specifies the C datatype of the keyword value and may have
any of the values listed in the description of the keyword reading
routines, above.  A NULL may be entered for the comment parameter, in
which case the  keyword comment field will be unmodified or left
blank.

The third routine is more primitive and simply writes the 80-character
{\tt card} record to the header.  It is the programmer's responsibility
in this case to ensure that the record conforms to all the FITS format
requirements for a header record.

The fourth routine modifies the comment string in an existing keyword,
and the last routine writes or updates the keyword units string for an
existing keyword.  (The units are recorded at the beginning of the
keyword comment field enclosed in square brackets).

\begin{verbatim}
___________________________________________________________________
int fits_write_comment(fitsfile *fptr, char *comment,  int *status)
int fits_write_history(fitsfile *fptr, char *history,  int *status)
int fits_write_date(fitsfile *fptr,  int *status)
\end{verbatim}

Write a {\tt COMMENT, HISTORY}, or {\tt DATE} keyword to the current
header.  The {\tt COMMENT} keyword is typically used to write a comment
about the file or the data.  The {\tt HISTORY} keyword is typically
used to provide information about the history of the processing
procedures that have been applied to the data.  The {\tt comment} or
{\tt history} string will be continued over multiple keywords if it is
more than 70 characters long.

The {\tt DATE} keyword is used to record the date and time that the
FITS file was created.  Note that this file creation date is usually
different from the date of the observation which obtained the data in
the FITS file.  The {\tt DATE} keyword value is a character string in
'yyyy-mm-ddThh:mm:ss' format. If a {\tt DATE} keyword already exists in
the header, then this routine will update the value with the current
system date.

\begin{verbatim}
___________________________________________________________________
int fits_delete_record(fitsfile *fptr, int keynum,  int *status)
int fits_delete_key(fitsfile *fptr, char *keyname,  int *status)
\end{verbatim}

Delete a keyword record. The first routine deletes a keyword at a
specified position (the first keyword is at position 1, not 0),
whereas the second routine deletes the named keyword.

\begin{verbatim}
_______________________________________________________________________
int fits_copy_header(fitsfile *infptr, fitsfile *outfptr,  int *status)
\end{verbatim}

Copy all the header keywords from the current HDU associated with
infptr to the current HDU associated with outfptr.  If the current
output HDU is not empty, then a new HDU will be appended to the output
file. The output HDU will then have the identical structure as the
input HDU, but will contain no data.

\newpage
% ===================================================================
\subsection{Utility Routines}

This section lists the most important CFITSIO general utility routines.

\begin{verbatim}
___________________________________________________________________
int fits_write_chksum( fitsfile *fptr, int *status)
int fits_verify_chksum(fitsfile *fptr, int *dataok, int *hduok, int *status)
\end{verbatim}

These routines  compute or validate the checksums for the currenrt
HDU.  The {\tt DATASUM} keyword is used to store the numerical value of
the 32-bit, 1's complement checksum for the data unit alone.  The {\tt
CHECKSUM} keyword is used to store the ASCII encoded COMPLEMENT of the
checksum for the entire HDU.  Storing the complement, rather than the
actual checksum, forces the checksum for the whole HDU to equal zero.
If the file has been modified since the checksums were computed, then
the HDU checksum will usually not equal zero.

The returned {\tt dataok} and {\tt hduok} parameters will have a value
= 1 if the data or HDU is verified correctly, a value = 0 if the
{\tt DATASUM} or {\tt CHECKSUM} keyword is not present, or value = -1 if the
computed checksum is not correct.


\begin{verbatim}
___________________________________________________________________
int fits_parse_value(char *card, char *value, char *comment, int *status)
int fits_get_keytype(char *value, char *dtype, int *status)
int fits_get_keyclass(char *card)
int fits_parse_template(char *template, char *card, int *keytype, int *status)

\end{verbatim}

{\tt fits\_parse\_value} parses the input 80-chararacter header keyword record, returning
the value (as a literal character string) and comment strings.  If the
keyword has no value (columns 9-10 not equal to '= '), then a null
value string is returned and the comment string is set equal to column
9 - 80 of the input string.

{\tt fits\_get\_keytype} parses the keyword value string to determine its
datatype.  {\tt dtype} returns with a value of 'C', 'L', 'I', 'F' or
'X', for character string, logical, integer, floating point, or
complex, respectively.

{\tt fits\_get\_keyclass} returns a classification code that indicates
the classification type of the input keyword record (e.g., a required
structural keyword, a TDIM keyword, a WCS keyword, a comment keyword,
etc.  See the CFITSIO Reference Guide for a list of the different
classification codes.

{\tt fits\_parse\_template} takes an input free format keyword template
string and returns a formatted 80*char record that satisfies all the
FITS requirements for a header keyword record.  The template should
generally contain 3 tokens: the keyword name, the keyword value, and
the keyword comment string.  The returned {\tt keytype} parameter
indicates whether the keyword is a COMMENT keyword or not.   See the
CFITSIO Reference Guide for more details.

\newpage
% ===================================================================
\section{CFITSIO File Names and Filters}

\subsection{Creating New Files}

When creating a new output file on magnetic disk  with {\tt
fits\_create\_file} the following features are supported.
\begin{itemize}
\item Overwriting, or 'Clobbering' an Existing File

If the filename is preceded by an exclamation
point (!) then if that file already exists it will be deleted prior to
creating the new FITS file.  Otherwise if there is an existing file
with the same name, CFITSIO will not overwrite the existing file and
will return an error status code.  Note  that the exclamation point is
a special UNIX character, so if it is used on the command line rather
than entered at a task prompt, it must be preceded by a backslash to
force the UNIX shell to pass it verbatim to the application program.

\item Compressed Output Files

If the output disk file name ends with the suffix '.gz', then CFITSIO
will compress the file using the gzip compression algorithm before
writing it to disk.  This can reduce the amount of disk space used by
the file.  Note that this feature requires that the uncompressed file
be constructed in memory before it is compressed and written to disk,
so it can fail if there is insufficient available memory.

One can also specify that any images written to the output file should
be compressed using the newly developed `tile-compression' algorithm by
appending `[compress]' to the name of the disk file (as in
{\tt myfile.fits[compress]}).   Refer to the CFITSIO User's Reference Guide
for more information about this new image compression format.

\item Using a Template to Create a New FITS File

The structure of any new FITS file that is to be created may be defined
in an ASCII template file.  If the name of the template file is
appended to the name of the FITS file itself, enclosed in parenthesis
(e.g., {\tt 'newfile.fits(template.txt)'}) then CFITSIO will create a
FITS file with that structure before opening it for the application to
use.  The template file basically defines the dimensions and data type
of the primary array and any IMAGE extensions, and the names and data
types of the columns in any ASCII or binary table extensions.  The
template file can also be used to define any optional keywords that
should be written in any of the HDU headers.  The image pixel values
and table entry values are all initialized to zero.  The application
program can then write actual data into the HDUs.  See the CFITSIO
Reference Guide for for a complete description of the template file
syntax.

\item Creating a Temporary Scratch File in Memory

It is sometimes useful to create a temporary output file when testing
an application program.  If the name of the file to be created is
specified as {\tt mem:} then CFITSIO will create the file in
memory where it will persist only until the program closes the file.
Use of this {\tt mem:} output file usually enables the program to run
faster, and of course the output file does not use up any disk space.


\end{itemize}

\subsection{Opening Existing Files}

When opening a file with {\tt fits\_open\_file}, CFITSIO can read a
variety of different input file formats and is not restricted to only
reading FITS format files from magnetic disk. The following types of
input files are all supported:

\begin{itemize}
\item FITS files compressed with {\tt zip, gzip} or {\tt compress}

If CFITSIO cannot find the specified file to open it will automatically
look for a file with the same rootname but with a {\tt .gz, .zip}, or
{\tt .Z} extension.  If it finds such a compressed file, it will
allocate a block of memory and uncompress the file into that memory
space.  The application program will then transparently open this
virtual FITS file in memory.  Compressed
files can only be opened with 'readonly', not 'readwrite' file access.

\item  FITS files on the internet, using {\tt ftp} or {\tt http} URLs

Simply provide the full URL as the name of the file that you want to
open.  For example,\linebreak {\tt
ftp://legacy.gsfc.nasa.gov/software/fitsio/c/testprog.std}\linebreak
will open the CFITSIO test FITS file that is located on the {\tt
legacy} machine.  These files can only be opened with 'readonly' file
access.

\item  FITS files on {\tt stdin} or {\tt stdout} file streams

If the name of the file to be opened is {\tt 'stdin'} or {\tt '-'} (a
single dash character) then CFITSIO will read the file from the
standard input stream.  Similarly, if the output file name is {\tt
'stdout'} or {\tt '-'}, then the file will be written to the standard
output stream.  In addition, if the output filename is {\tt
'stdout.gz'} or {\tt '-.gz'} then it will be gzip compressed before
being written to stdout.  This mechanism can be used to pipe FITS files
from one task to another without having to write an intermediary FITS
file on magnetic disk.

\item FITS files that exist only in memory, or shared memory.

In some applications, such as real time data acquisition, you may want
to have one process write a FITS file into a certain section of
computer memory, and then be able to open that file in memory with
another process.  There is a specialized CFITSIO open routine called
{\tt fits\_open\_memfile} that can be used for this purpose.  See the
``CFITSIO User's Reference Guide'' for more details.

\item  IRAF format images (with {\tt .imh} file extensions)

CFITSIO supports reading IRAF format images by converting them on the
fly into FITS images in memory.  The application program then reads
this virtual FITS format image in memory.  There is currently no
support for writing IRAF format images, or for reading or writing IRAF
tables.

\item Image arrays in raw binary format

If the input file is a raw binary data array, then CFITSIO will convert
it on the fly into a virtual FITS image with the basic set of required
header keywords before it is opened by the application program.  In
this case the data type and dimensions of the image must be specified
in square brackets following the filename (e.g. {\tt
rawfile.dat[ib512,512]}). The first character inside the brackets
defines the datatype of the array:

\begin{verbatim}
     b         8-bit unsigned byte
     i        16-bit signed integer
     u        16-bit unsigned integer
     j        32-bit signed integer
     r or f   32-bit floating point
     d        64-bit floating point
\end{verbatim}
An optional second character specifies the byte order of the array
values: b or B indicates big endian (as in FITS files and the native
format of SUN UNIX workstations and Mac PCs) and l or L indicates
little endian (native format of DEC OSF workstations and IBM PCs).  If
this character is omitted then the array is assumed to have the native
byte order of the local machine.  These datatype characters are then
followed by a series of one or more integer values separated by commas
which define the size of each dimension of the raw array.  Arrays with
up to 5 dimensions are currently supported.  

Finally, a byte offset to the position of the first pixel in the data
file may be specified by separating it with a ':' from the last
dimension value.  If omitted, it is assumed that the offset = 0.  This
parameter may be used to skip over any header information in the file
that precedes the binary data.  Further examples:
 
\begin{verbatim}
  raw.dat[b10000]          1-dimensional 10000 pixel byte array
  raw.dat[rb400,400,12]    3-dimensional floating point big-endian array
  img.fits[ib512,512:2880] reads the 512 x 512 short integer array in a
                           FITS file, skipping over the 2880 byte header
\end{verbatim}

\end{itemize}
\newpage

\subsection{Image Filtering}

\subsubsection{Extracting a subsection of an image}

When specifying the name of an image to be opened, you can select a
rectangular subsection of the image to be extracted and opened by the
application program.  The application program then opens a virtual
image that only contains the pixels within the specified subsection.
To do this, specify the the range of pixels (start:end) along each axis
to be extracted from the original image enclosed in square brackets.
You can also specify an optional pixel increment (start:end:step) for
each axis of the input image.  A pixel step = 1 will be assumed if it
is not specified.  If the starting pixel is larger then the end pixel,
then the image will be flipped (producing a mirror image) along that
dimension.  An asterisk, '*', may be used to specify the entire range
of an axis, and '-*' will flip the entire axis.  In the following
examples, assume that {\tt myfile.fits} contains a 512 x 512 pixel 2D
image.

\begin{verbatim}
  myfile.fits[201:210, 251:260] - opens a 10 x 10 pixel subimage.

  myfile.fits[*, 512:257] - opens a 512 x 256 image consisting of
	      all the columns in the input image, but only rows 257
	      through 512.  The image will be flipped along the Y axis
	      since the starting row is greater than the ending
	      row.

  myfile.fits[*:2, 512:257:2] - creates a 256 x 128 pixel image.
	      Similar to the previous example, but only every other row
	      and column is read from the input image.

  myfile.fits[-*, *] - creates an image containing all the rows and
	      columns in the input image, but flips it along the X
	      axis.
\end{verbatim}

If the array to be opened is in an Image extension, and not in the
primary array of the file, then you need to specify the extension
name or number in square brackets before giving the subsection range,
as in {\tt  myfile.fits[1][-*, *]} to read the image in the
first extension in the file.

\subsubsection{Create an Image by Binning Table Columns}

You can also create and open a virtual image by binning the values in a
pair of columns of a FITS table (in other words, create a 2-D histogram
of the values in the 2 columns).  This technique is often used in X-ray
astronomy where each detected X-ray photon during an observation is
recorded in a FITS table.  There are typically 2 columns in the table
called  {\tt X} and {\tt Y} which record the pixel location of that
event in a virtual 2D image.  To create an image from this table, one
just scans the X and Y columns and counts up how many photons were
recorded in each pixel of the image.  When table binning is specified,
CFITSIO creates a temporary FITS primary array in memory by computing
the histogram of the values in the specified columns.  After the
histogram is computed the original FITS file containing the table is
closed and the temporary FITS primary array is opened and passed to the
application program.  Thus, the application program never sees the
original FITS table and only sees the image in the new temporary file
(which has no extensions).

The table binning specifier is enclosed in square brackets following
the root filename and table extension name or number and begins with
the keyword 'bin', as in: \newline  
{\tt 'myfile.fits[events][bin (X,Y)]'}. In
this case, the X and Y columns in the 'events' table extension are
binned up to create the image.  The size of the image is usually
determined by the {\tt TLMINn} and {\tt TLMAXn} header keywords which
give the minimum and maximum allowed pixel values in the columns.  For
instance if {\tt TLMINn = 1} and {\tt TLMAXn = 4096} for both columns, this would
generate a 4096 x 4096 pixel image by default.  This is rather large,
so you can also specify a pixel binning factor to reduce the image
size.  For example specifying ,  {\tt '[bin (X,Y) = 16]'} will use a
binning factor of 16, which will produce a 256 x 256 pixel image in the
previous example.

If the TLMIN and TLMAX keywords don't exist, or you want to override
their values,  you can specify the image range and binning factor
directly, as in {\tt '[bin X = 1:4096:16, Y=1:4096:16]'}.  You can also
specify the datatype of the created image by appending a b, i, j, r, or
d (for 8-bit byte, 16-bit integers, 32-bit integer, 32-bit floating
points, or 64-bit double precision floating point, respectively)  to
the 'bin' keyword (e.g. {\tt '[binr (X,Y)]'} creates a floating point
image).  If the datatype is not specified then a 32-bit integer image
will be created by default.

If the column name is not specified, then CFITSIO will first try to use
the 'preferred column' as specified by the CPREF keyword if it exists
(e.g., 'CPREF = 'DETX,DETY'), otherwise column names 'X', 'Y' will be
assumed for the 2 axes.

Note that this binning specifier is not restricted to only 2D images
and can be used to create 1D, 3D, or 4D images as well.  It is also
possible to specify a weighting factor that is applied during the
binning.  Please refer to the ``CFITSIO User's Reference Guide'' for
more details on these advanced features.
\newpage

\subsection{Table Filtering}

\subsubsection{Column and Keyword Filtering}

The column or keyword filtering specifier is used to modify the
column structure and/or the header keywords in the HDU that was
selected with the previous HDU location specifier.   It can
be used to perform the following types of operations. 

\begin{itemize}
\item
Append a new column to a table by giving the column name, optionally
followed by the datatype in parentheses, followed by an equals sign and
the arithmetic  expression to be used to compute the value.  The
datatype is specified using the same syntax that is allowed for the
value of the FITS TFORMn keyword (e.g., 'I', 'J', 'E', 'D', etc. for
binary tables, and 'I8', F12.3', 'E20.12', etc.  for ASCII tables).  If
the datatype is not specified then a default datatype will be chosen
depending on the expression.

\item
Create a new header keyword by giving the keyword name, preceded by a
pound sign '\#', followed by an equals sign and an arithmetic
expression for the value of the keyword.  The expression may be a
function of other header keyword values.  The comment string for the
keyword may be specified in parentheses immediately following the
keyword name.

\item
Overwrite the values in an existing column or keyword by giving the
name followed by an equals sign and an arithmetic expression.

\item
Select a set of columns to be included in the filtered file by listing
the column names separated with semi-colons.   Wild card characters may
be used in the column names to match multiple columns.  Any other
columns in the input table will not appear in the filtered file.

\item
Delete a column or keyword by listing the name preceded by a minus sign
or an exclamation mark (!)

\item
Rename an existing column or keyword with the syntax 'NewName ==
OldName'.

\end{itemize}

The column filtering specifier is enclosed in square brackets and
begins with the string 'col'.   Multiple operations can be performed
by separating them with semi-colons.  For  complex  or commonly used
operations,  you can write the column filter to a text file, and then
use it by giving the name of the text file, preceded by a '@'
character.

Some examples:

\begin{verbatim}
  [col PI=PHA * 1.1 + 0.2]      - creates new PI column from PHA values

  [col rate = counts/exposure]  - creates or overwrites the rate column by
                                  dividing the counts column by the
                                  EXPOSURE keyword value.

  [col TIME; X; Y]              - only the listed columns will appear
                                  in the filtered file

  [col Time;*raw]               - include the Time column and any other
                                  columns whose name ends with 'raw'.

  [col -TIME; Good == STATUS]   - deletes the TIME column and
                                  renames the STATUS column to GOOD

  [col @colfilt.txt]            - uses the filtering expression in
                                  the colfilt.txt text file
\end{verbatim}

The original file is not changed by this filtering operation, and
instead the modifications are made on a temporary copy of the input
FITS file (usually in memory), which includes a copy of all the other
HDUs in the input file.   The original input file is closed and the
application program opens the filtered copy of the file.

\subsubsection{Row Filtering}

The row filter is used to select a subset of the rows from a table
based on a boolean expression.  A temporary new FITS file is created on
the fly (usually in memory) which contains only those rows for which
the row filter expression evaluates to true (i.e., not equal to zero).
The primary array and any other extensions in the input file are also
copied to the temporary file.  The original FITS file is closed and the
new temporary file is then opened by the application program.

The row filter expression is enclosed in square brackets following the
file name and extension name.  For example, {\tt
'file.fits[events][GRADE==50]'}  selects only those rows in the EVENTS
table where the GRADE column value is equal to 50).

The row filtering  expression can be an arbitrarily  complex series of
operations performed  on constants,  keyword values,  and column data
taken from the specified FITS TABLE extension.  The expression 
also can be written into a text file and then used by giving the
filename preceded by a '@' character, as in
{\tt '[@rowfilt.txt]'}.

Keyword and column data  are referenced by   name.  Any  string of
characters not surrounded by    quotes (ie, a constant  string)   or
followed by   an open parentheses (ie,   a  function name)   will be
initially interpreted   as a column  name and  its contents for the
current row inserted into the expression.  If no such column exists,
a keyword of that  name will be searched for  and its value used, if
found.  To force the  name to be  interpreted as a keyword (in case
there is both a column and keyword with the  same name), precede the
keyword name with a single pound sign, '\#', as in {\tt \#NAXIS2}.  Due to
the generalities of FITS column and  keyword names, if the column or
keyword name  contains a space or a  character which might appear as
an arithmetic  term then inclose  the  name in '\$'  characters as in
{\tt \$MAX PHA\$} or {\tt \#\$MAX-PHA\$}.  The names are case insensitive.

To access a table entry in a row other  than the current one, follow
the  column's name  with  a row  offset  within  curly  braces.  For
example, {\tt'PHA\{-3\}'} will evaluate to the value  of column PHA, 3 rows
above  the  row currently  being processed.   One  cannot specify an
absolute row number, only a relative offset.  Rows that fall outside
the table will be treated as undefined, or NULLs.

Boolean   operators can be  used in  the expression  in either their
Fortran or C forms.  The following boolean operators are available:

\begin{verbatim}
    "equal"         .eq. .EQ. ==  "not equal"          .ne.  .NE.  !=
    "less than"     .lt. .LT. <   "less than/equal"    .le.  .LE.  <= =<
    "greater than"  .gt. .GT. >   "greater than/equal" .ge.  .GE.  >= =>
    "or"            .or. .OR. ||  "and"                .and. .AND. &&
    "negation"     .not. .NOT. !  "approx. equal(1e-7)"  ~
\end{verbatim}

Note  that the exclamation point,  '!', is a special UNIX character, so
if it is used  on the command line rather than entered at a task
prompt, it must be  preceded by a backslash to force the UNIX shell to
ignore it.

The expression may  also include arithmetic operators and functions.
Trigonometric  functions use  radians,  not degrees.  The  following
arithmetic  operators and  functions  can be  used in the expression
(function names are case insensitive):

 
\begin{verbatim}
    "addition"           +          "subtraction"          -
    "multiplication"     *          "division"             /
    "negation"           -          "exponentiation"       **   ^
    "absolute value"     abs(x)     "cosine"               cos(x)
    "sine"               sin(x)     "tangent"              tan(x)
    "arc cosine"         arccos(x)  "arc sine"             arcsin(x)
    "arc tangent"        arctan(x)  "arc tangent"          arctan2(x,y)
    "exponential"        exp(x)     "square root"          sqrt(x)
    "natural log"        log(x)     "common log"           log10(x)
    "modulus"            i % j      "random # [0.0,1.0)"   random()
    "minimum"            min(x,y)   "maximum"              max(x,y)
    "if-then-else"       b?x:y
\end{verbatim}


The  following  type  casting  operators  are  available,  where the
inclosing parentheses are required and taken  from  the  C  language
usage. Also, the integer to real casts values to double precision:

\begin{verbatim}
                "real to integer"    (int) x     (INT) x
                "integer to real"    (float) i   (FLOAT) i
\end{verbatim}


Several constants are built in  for  use  in  numerical
expressions:

 
\begin{verbatim}
        #pi              3.1415...      #e             2.7182...
        #deg             #pi/180        #row           current row number
        #null         undefined value   #snull         undefined string
\end{verbatim}

A  string constant must  be enclosed  in quotes  as in  'Crab'.  The
"null" constants  are useful for conditionally  setting table values to
a NULL, or undefined, value (For example,  {\tt "col1==-99 ? \#NULL :
col1"}).

There is also a function for testing if  two  values  are  close  to
each  other,  i.e.,  if  they are "near" each other to within a user
specified tolerance. The  arguments,  {\tt value\_1}  and  {\tt value\_2}  can  be
integer  or  real  and  represent  the two values who's proximity is
being tested to be within the specified tolerance, also  an  integer
or real:

\begin{verbatim}
                    near(value_1, value_2, tolerance)
\end{verbatim}

When  a  NULL, or undefined, value is encountered in the FITS table,
the expression will evaluate to NULL unless the undefined  value  is
not   actually   required  for  evaluation,  e.g. "TRUE  .or.  NULL"
evaluates to TRUE. The  following  two  functions  allow  some  NULL
detection  and  handling:  

\begin{verbatim}
                   ISNULL(x)
                   DEFNULL(x,y)
\end{verbatim}

The former returns a boolean value of TRUE if the  argument  x  is
NULL.   The later  "defines"  a  value  to  be  substituted  for NULL
values; it returns the value of x if x is not NULL, otherwise  it
returns  the value of y.

Bit  masks can be used to select out rows from bit columns ({\tt TFORMn =
\#X}) in FITS files. To represent the mask,  binary,  octal,  and  hex
formats are allowed:
 
\begin{verbatim}
                 binary:   b0110xx1010000101xxxx0001
                 octal:    o720x1 -> (b111010000xxx001)
                 hex:      h0FxD  -> (b00001111xxxx1101)
\end{verbatim}

In  all  the  representations, an x or X is allowed in the mask as a
wild card. Note that the x represents a  different  number  of  wild
card  bits  in  each  representation.  All  representations are case
insensitive.

To construct the boolean expression using the mask  as  the  boolean
equal  operator  described above on a bit table column. For example,
if you had a 7 bit column named flags in a  FITS  table  and  wanted
all  rows  having  the bit pattern 0010011, the selection expression
would be:

 
\begin{verbatim}
                            flags == b0010011
    or
                            flags .eq. b10011
\end{verbatim}

It is also possible to test if a range of bits is  less  than,  less
than  equal,  greater  than  and  greater than equal to a particular
boolean value:

 
\begin{verbatim}
                            flags <= bxxx010xx
                            flags .gt. bxxx100xx
                            flags .le. b1xxxxxxx
\end{verbatim}

Notice the use of the x bit value to limit the range of  bits  being
compared.

It  is  not necessary to specify the leading (most significant) zero
(0) bits in the mask, as shown in the second expression above.

Bit wise AND, OR and NOT operations are  also  possible  on  two  or
more  bit  fields  using  the  '\&'(AND),  '$|$'(OR),  and the '!'(NOT)
operators. All of these operators result in a bit  field  which  can
then be used with the equal operator. For example:

 
\begin{verbatim}
                          (!flags) == b1101100
                          (flags & b1000001) == bx000001
\end{verbatim}

Bit  fields can be appended as well using the '+' operator.  Strings
can be concatenated this way, too.

\subsubsection{Good Time Interval Filtering}

    A common filtering method involves selecting rows which have a time
    value which lies within what is called a Good Time Interval or GTI.
    The time intervals are defined in a separate FITS table extension
    which contains 2 columns giving the start and stop time of each
    good interval.  The filtering operation accepts only those rows of
    the input table which have an associated time which falls within
    one of the time intervals defined in the GTI extension. A high
    level function, gtifilter(a,b,c,d), is available which evaluates
    each row of the input table  and returns TRUE  or FALSE depending
    whether the row is inside or outside the  good time interval.  The
    syntax is
 
\begin{verbatim}
      gtifilter( [ "gtifile" [, expr [, "STARTCOL", "STOPCOL" ] ] ] )
\end{verbatim}
    where  each "[]" demarks optional parameters.  Note that  the quotes
    around the gtifile and START/STOP column are required.  Either single
    or double quote characters may be used.  The gtifile,
    if specified,  can be blank  ("") which will  mean to use  the first
    extension  with   the name "*GTI*"  in   the current  file,  a plain
    extension  specifier (eg, "+2",  "[2]", or "[STDGTI]") which will be
    used  to  select  an extension  in  the current  file, or  a regular
    filename with or without an extension  specifier which in the latter
    case  will mean to  use the first  extension  with an extension name
    "*GTI*".  Expr can be   any arithmetic expression, including  simply
    the time  column  name.  A  vector  time expression  will  produce a
    vector boolean  result.  STARTCOL and  STOPCOL are the  names of the
    START/STOP   columns in the    GTI extension.  If   one  of them  is
    specified, they both  must be.

    In  its  simplest form, no parameters need to be provided -- default
    values will be used.  The expression {\tt "gtifilter()"} is equivalent to
 
\begin{verbatim}
       gtifilter( "", TIME, "*START*", "*STOP*" )
\end{verbatim}
    This will search the current file for a GTI  extension,  filter  the
    TIME  column in the current table, using START/STOP times taken from
    columns in the GTI  extension  with  names  containing  the  strings
    "START"  and "STOP".  The wildcards ('*') allow slight variations in
    naming conventions  such  as  "TSTART"  or  "STARTTIME".   The  same
    default  values  apply for unspecified parameters when the first one
    or  two  parameters  are  specified.   The  function   automatically
    searches   for   TIMEZERO/I/F   keywords  in  the  current  and  GTI
    extensions, applying a relative time offset, if necessary.

\subsubsection{Spatial Region Filtering}

    Another common  filtering method selects rows based on whether the
    spatial position associated with each row is located within a given
    2-dimensional region.  The syntax for this high-level filter is
 
\begin{verbatim}
       regfilter( "regfilename" [ , Xexpr, Yexpr [ , "wcs cols" ] ] )
\end{verbatim}
    where each "[ ]" demarks optional parameters. The region file name
    is required and must be  enclosed in quotes.  The remaining
    parameters are optional.  The region file is an ASCII text file
    which contains a list of one or more geometric shapes (circle,
    ellipse, box, etc.) which defines a region on the celestial sphere
    or an area within a particular 2D image.  The region file is
    typically generated using an image display program such as fv/POW
    (distribute by the HEASARC), or ds9 (distributed by the Smithsonian
    Astrophysical Observatory).  Users should refer to the documentation
    provided with these programs for more details on the syntax used in
    the region files.

    In its simpliest form, (e.g., {\tt regfilter("region.reg")} ) the
    coordinates in the default 'X' and 'Y' columns will be used to
    determine if each row is inside or outside the area specified in
    the region file.  Alternate position column names, or expressions,
    may be entered if needed, as in
 
\begin{verbatim}
        regfilter("region.reg", XPOS, YPOS)
\end{verbatim}
    Region filtering can be applied most unambiguously if the positions
    in the region file and in the table to be filtered are both give in
    terms of absolute celestial coordinate units.  In this case the
    locations and sizes of the geometric shapes in the region file are
    specified in angular units on the sky (e.g., positions given in
    R.A. and Dec.  and sizes in arcseconds or arcminutes).  Similarly,
    each row of the filtered table will have a celestial coordinate
    associated with it.  This association is usually implemented using
    a set of so-called 'World Coordinate System' (or WCS) FITS keywords
    that define the coordinate transformation that must be applied to
    the values in the 'X' and 'Y' columns to calculate the coordinate.

    Alternatively, one can perform spatial filtering using unitless
    'pixel' coordinates for the regions and row positions.  In this
    case the user must be careful to ensure that the positions in the 2
    files are self-consistent.  A typical problem is that the region
    file may be generated using a binned image, but the unbinned
    coordinates are given in the event table.  The ROSAT events files,
    for example, have X and Y pixel coordinates that range from 1 -
    15360.  These coordinates are typically binned by a factor of 32 to
    produce a 480x480 pixel image.  If one then uses a region file
    generated from this image (in image pixel units) to filter the
    ROSAT events file, then the X and Y column values must be converted
    to corresponding pixel units as in:
 
\begin{verbatim}
        regfilter("rosat.reg", X/32.+.5, Y/32.+.5)
\end{verbatim}
    Note that this binning conversion is not necessary if the region
    file is specified using celestial coordinate units instead of pixel
    units because CFITSIO is then able to directly compare the
    celestial coordinate of each row in the table with the celestial
    coordinates in the region file without having to know anything
    about how the image may have been binned.

    The last "wcs cols" parameter should rarely be needed. If supplied,
    this  string contains the names of the 2 columns (space or comma
    separated) which have the associated WCS keywords. If not supplied,
    the filter  will scan the X  and Y expressions for column names.
    If only one is found in each  expression, those columns will be
    used, otherwise an error will be returned.

    These region shapes are supported (names are case insensitive):
 
\begin{verbatim}
       Point         ( X1, Y1 )               <- One pixel square region
       Line          ( X1, Y1, X2, Y2 )       <- One pixel wide region
       Polygon       ( X1, Y1, X2, Y2, ... )  <- Rest are interiors with
       Rectangle     ( X1, Y1, X2, Y2, A )       | boundaries considered
       Box           ( Xc, Yc, Wdth, Hght, A )   V within the region
       Diamond       ( Xc, Yc, Wdth, Hght, A )
       Circle        ( Xc, Yc, R )
       Annulus       ( Xc, Yc, Rin, Rout )
       Ellipse       ( Xc, Yc, Rx, Ry, A )
       Elliptannulus ( Xc, Yc, Rinx, Riny, Routx, Routy, Ain, Aout )
       Sector        ( Xc, Yc, Amin, Amax )
\end{verbatim}
    where (Xc,Yc) is  the coordinate of  the shape's center; (X\#,Y\#) are
    the coordinates  of the shape's edges;  Rxxx are the shapes' various
    Radii or semimajor/minor  axes; and Axxx  are the angles of rotation
    (or bounding angles for Sector) in degrees.  For rotated shapes, the
    rotation angle  can  be left  off, indicating  no rotation.   Common
    alternate  names for the regions  can also be  used: rotbox = box;
    rotrectangle = rectangle;  (rot)rhombus = (rot)diamond;  and pie
    = sector.  When a  shape's name is  preceded by a minus sign, '-',
    the defined region  is instead the area  *outside* its boundary (ie,
    the region is inverted).  All the shapes within a single region
    file are OR'd together to create the region, and the order is
    significant. The overall way of looking at region files is that if
    the first region is an excluded region then a dummy included region
    of the whole detector is inserted in the front. Then each region
    specification as it is processed overrides any selections inside of
    that region specified by previous regions. Another way of thinking
    about this is that if a previous excluded region is completely
    inside of a subsequent included region the excluded region is
    ignored.

    The positional coordinates may be given either in pixel units,
    decimal degrees or hh:mm:ss.s, dd:mm:ss.s units.  The shape sizes
    may be given in pixels, degrees, arcminutes, or arcseconds.  Look
    at examples of region file produced by fv/POW or ds9 for further
    details of the region file format.

\subsubsection{Example Row Filters}
 
\begin{verbatim}
    [double && mag <= 5.0]        -  Extract all double stars brighter
                                     than  fifth magnitude 

    [#row >= 125 && #row <= 175]   - Extract row numbers 125 through 175

    [abs(sin(theta * #deg)) < 0.5] - Extract all rows having the
                                     absolute value of the sine of theta
                                     less  than a half where the angles
                                     are tabulated in degrees

    [@rowFilter.txt]               - Extract rows using the expression
                                     contained within the text file
                                     rowFilter.txt

    [gtifilter()]                  - Search the current file for a GTI
				     extension,  filter  the TIME
				     column in the current table, using
				     START/STOP times taken from
				     columns in the GTI  extension

    [regfilter("pow.reg")]         - Extract rows which have a coordinate
                                     (as given in the X and Y columns)
                                     within the spatial region specified
                                     in the pow.reg region file.
\end{verbatim}

\newpage
\subsection{Combined Filtering Examples}

The previous sections described all the individual types of filters
that may be applied to the input file.  In this section we show
examples which combine several different filters at once.  These
examples all use the {\tt fitscopy} program that is distributed with
the CFITSIO code.  It simply copies the input file to the output file.

\begin{verbatim}
fitscopy rosat.fit out.fit
\end{verbatim}

This trivial example simply makes an identical copy of the input
rosat.fit file without any filtering.

\begin{verbatim}
fitscopy 'rosat.fit[events][col Time;X;Y][#row < 1000]' out.fit
\end{verbatim}

The output file contains only the Time, X, and Y columns, and only
the first 999 rows from the 'EVENTS' table extension of the input file.
All the other HDUs in the input file are copied to the output file
without any modification.

\begin{verbatim}
fitscopy 'rosat.fit[events][PI < 50][bin (Xdet,Ydet) = 16]' image.fit
\end{verbatim}

This creates an output image by binning the Xdet and Ydet columns of
the events table with a pixel binning factor of 16.  Only the rows
which have a PI energy less than 50 are used to construct this image.
The output image file contains a primary array image without any
extensions.

\begin{verbatim}
fitscopy 'rosat.fit[events][gtifilter() && regfilter("pow.reg")]' out.fit
\end{verbatim}

The filtering expression in this example uses the {\tt gtifilter}
function to test whether the TIME column value in each row is within
one of the Good Time Intervals defined in the GTI extension in the same
input file, and also uses the {\tt regfilter} function to test if the
position associated with each row (derived by default from the values
in the X and Y columns of the events table) is located within the area
defined in the {\tt pow.reg} text region file (which was previously
created with the {\tt fv/POW} image display program).  Only the rows
which satisfy both tests are copied to the output table.

\begin{verbatim}
fitscopy 'r.fit[evt][PI<50]' stdout | fitscopy stdin[evt][col X,Y] out.fit
\end{verbatim}

In this somewhat convoluted example, fitscopy is used to first select
the rows from the evt extension which have PI less than 50 and write the
resulting table out to the stdout stream.  This is piped to a 2nd
instance of fitscopy (with the Unix `$|$' pipe command) which reads that
filtered FITS file from the stdin stream and copies only the X and Y
columns from the evt table to the output file.

\begin{verbatim}
fitscopy 'r.fit[evt][col RAD=sqrt((X-#XCEN)**2+(Y-#YCEN)**2)][rad<100]' out.fit
\end{verbatim}

This example first creates a new column called RAD which gives the
distance between the X,Y coordinate of each event and the coordinate
defined by the XCEN and YCEN keywords in the header.  Then, only those
rows which have a distance less than 100 are copied to the output
table.  In other words, only the events which are located within 100
pixel units from the (XCEN, YCEN) coordinate are copied to the output
table.

\begin{verbatim}
fitscopy 'ftp://heasarc.gsfc.nasa.gov/rosat.fit[events][bin (X,Y)=16]' img.fit
\end{verbatim}

This example bins the X and Y columns of the hypothetical ROSAT file 
at the HEASARC ftp site to create the output image.

\begin{verbatim}
fitscopy 'raw.fit[i512,512][101:110,51:60]' image.fit
\end{verbatim}

This example converts the 512 x 512 pixel raw binary 16-bit integer
image to a FITS file and copies a 10 x 10 pixel subimage from it to the
output FITS image.

\newpage
\section{CFITSIO Error Status Codes}

The following table lists all the error status codes used by CFITSIO.
Programmers are encouraged to use the symbolic mnemonics (defined in
the file fitsio.h) rather than the actual integer status values to
improve the readability of their code.

\begin{verbatim}
 Symbolic Const    Value     Meaning
 --------------    -----  -----------------------------------------
                     0    OK, no error
 SAME_FILE         101    input and output files are the same
 TOO_MANY_FILES    103    tried to open too many FITS files at once
 FILE_NOT_OPENED   104    could not open the named file
 FILE_NOT_CREATED  105    could not create the named file
 WRITE_ERROR       106    error writing to FITS file
 END_OF_FILE       107    tried to move past end of file
 READ_ERROR        108    error reading from FITS file
 FILE_NOT_CLOSED   110    could not close the file
 ARRAY_TOO_BIG     111    array dimensions exceed internal limit
 READONLY_FILE     112    Cannot write to readonly file
 MEMORY_ALLOCATION 113    Could not allocate memory
 BAD_FILEPTR       114    invalid fitsfile pointer
 NULL_INPUT_PTR    115    NULL input pointer to routine 
 SEEK_ERROR        116    error seeking position in file 

 BAD_URL_PREFIX     121   invalid URL prefix on file name 
 TOO_MANY_DRIVERS   122   tried to register too many IO drivers 
 DRIVER_INIT_FAILED 123   driver initialization failed 
 NO_MATCHING_DRIVER 124   matching driver is not registered 
 URL_PARSE_ERROR    125   failed to parse input file URL

 SHARED_BADARG     151    bad argument in shared memory driver
 SHARED_NULPTR     152    null pointer passed as an argument
 SHARED_TABFULL    153    no more free shared memory handles
 SHARED_NOTINIT    154    shared memory driver is not initialized
 SHARED_IPCERR     155    IPC error returned by a system call
 SHARED_NOMEM      156    no memory in shared memory driver
 SHARED_AGAIN      157    resource deadlock would occur
 SHARED_NOFILE     158    attempt to open/create lock file failed
 SHARED_NORESIZE   159    shared memory block cannot be resized at the moment

 HEADER_NOT_EMPTY  201    header already contains keywords
 KEY_NO_EXIST      202    keyword not found in header
 KEY_OUT_BOUNDS    203    keyword record number is out of bounds
 VALUE_UNDEFINED   204    keyword value field is blank 
 NO_QUOTE          205    string is missing the closing quote
 BAD_KEYCHAR       207    illegal character in keyword name or card
 BAD_ORDER         208    required keywords out of order
 NOT_POS_INT       209    keyword value is not a positive integer
 NO_END            210    couldn't find END keyword
 BAD_BITPIX        211    illegal BITPIX keyword value
 BAD_NAXIS         212    illegal NAXIS keyword value
 BAD_NAXES         213    illegal NAXISn keyword value
 BAD_PCOUNT        214    illegal PCOUNT keyword value
 BAD_GCOUNT        215    illegal GCOUNT keyword value
 BAD_TFIELDS       216    illegal TFIELDS keyword value
 NEG_WIDTH         217    negative table row size
 NEG_ROWS          218    negative number of rows in table
 COL_NOT_FOUND     219    column with this name not found in table
 BAD_SIMPLE        220    illegal value of SIMPLE keyword
 NO_SIMPLE         221    Primary array doesn't start with SIMPLE
 NO_BITPIX         222    Second keyword not BITPIX
 NO_NAXIS          223    Third keyword not NAXIS
 NO_NAXES          224    Couldn't find all the NAXISn keywords
 NO_XTENSION       225    HDU doesn't start with XTENSION keyword
 NOT_ATABLE        226    the CHDU is not an ASCII table extension
 NOT_BTABLE        227    the CHDU is not a binary table extension
 NO_PCOUNT         228    couldn't find PCOUNT keyword
 NO_GCOUNT         229    couldn't find GCOUNT keyword
 NO_TFIELDS        230    couldn't find TFIELDS keyword
 NO_TBCOL          231    couldn't find TBCOLn keyword
 NO_TFORM          232    couldn't find TFORMn keyword
 NOT_IMAGE         233    the CHDU is not an IMAGE extension
 BAD_TBCOL         234    TBCOLn keyword value < 0 or > rowlength
 NOT_TABLE         235    the CHDU is not a table
 COL_TOO_WIDE      236    column is too wide to fit in table
 COL_NOT_UNIQUE    237    more than 1 column name matches template
 BAD_ROW_WIDTH     241    sum of column widths not = NAXIS1
 UNKNOWN_EXT       251    unrecognizable FITS extension type
 UNKNOWN_REC       252    unknown record; 1st keyword not SIMPLE or XTENSION
 END_JUNK          253    END keyword is not blank
 BAD_HEADER_FILL   254    Header fill area contains non-blank chars
 BAD_DATA_FILL     255    Illegal data fill bytes (not zero or blank)
 BAD_TFORM         261    illegal TFORM format code
 BAD_TFORM_DTYPE   262    unrecognizable TFORM datatype code
 BAD_TDIM          263    illegal TDIMn keyword value
 BAD_HEAP_PTR      264    invalid BINTABLE heap pointer is out of range

 BAD_HDU_NUM       301    HDU number < 1 or > MAXHDU
 BAD_COL_NUM       302    column number < 1 or > tfields
 NEG_FILE_POS      304    tried to move to negative byte location in file
 NEG_BYTES         306    tried to read or write negative number of bytes
 BAD_ROW_NUM       307    illegal starting row number in table
 BAD_ELEM_NUM      308    illegal starting element number in vector
 NOT_ASCII_COL     309    this is not an ASCII string column
 NOT_LOGICAL_COL   310    this is not a logical datatype column
 BAD_ATABLE_FORMAT 311    ASCII table column has wrong format
 BAD_BTABLE_FORMAT 312    Binary table column has wrong format
 NO_NULL           314    null value has not been defined
 NOT_VARI_LEN      317    this is not a variable length column
 BAD_DIMEN         320    illegal number of dimensions in array
 BAD_PIX_NUM       321    first pixel number greater than last pixel
 ZERO_SCALE        322    illegal BSCALE or TSCALn keyword = 0
 NEG_AXIS          323    illegal axis length < 1

 NOT_GROUP_TABLE       340   Grouping function error
 HDU_ALREADY_MEMBER    341
 MEMBER_NOT_FOUND      342
 GROUP_NOT_FOUND       343
 BAD_GROUP_ID          344
 TOO_MANY_HDUS_TRACKED 345
 HDU_ALREADY_TRACKED   346
 BAD_OPTION            347
 IDENTICAL_POINTERS    348
 BAD_GROUP_ATTACH      349
 BAD_GROUP_DETACH      350

 NGP_NO_MEMORY         360     malloc failed
 NGP_READ_ERR          361     read error from file
 NGP_NUL_PTR           362     null pointer passed as an argument.
                                 Passing null pointer as a name of
                                 template file raises this error
 NGP_EMPTY_CURLINE     363     line read seems to be empty (used
                                 internally)
 NGP_UNREAD_QUEUE_FULL 364     cannot unread more then 1 line (or single
                                 line twice)
 NGP_INC_NESTING       365     too deep include file nesting (infinite
                                 loop, template includes itself ?)
 NGP_ERR_FOPEN         366     fopen() failed, cannot open template file
 NGP_EOF               367     end of file encountered and not expected
 NGP_BAD_ARG           368     bad arguments passed. Usually means
                                 internal parser error. Should not happen
 NGP_TOKEN_NOT_EXPECT  369     token not expected here

 BAD_I2C           401    bad int to formatted string conversion
 BAD_F2C           402    bad float to formatted string conversion
 BAD_INTKEY        403    can't interpret keyword value as integer
 BAD_LOGICALKEY    404    can't interpret keyword value as logical
 BAD_FLOATKEY      405    can't interpret keyword value as float
 BAD_DOUBLEKEY     406    can't interpret keyword value as double
 BAD_C2I           407    bad formatted string to int conversion
 BAD_C2F           408    bad formatted string to float conversion
 BAD_C2D           409    bad formatted string to double conversion
 BAD_DATATYPE      410    illegal datatype code value
 BAD_DECIM         411    bad number of decimal places specified
 NUM_OVERFLOW      412    overflow during datatype conversion
 DATA_COMPRESSION_ERR   413  error compressing image 
 DATA_DECOMPRESSION_ERR 414  error uncompressing image 

 BAD_DATE          420    error in date or time conversion 

 PARSE_SYNTAX_ERR  431    syntax error in parser expression 
 PARSE_BAD_TYPE    432    expression did not evaluate to desired type 
 PARSE_LRG_VECTOR  433    vector result too large to return in array 
 PARSE_NO_OUTPUT   434    data parser failed not sent an out column 
 PARSE_BAD_COL     435    bad data encounter while parsing column 
 PARSE_BAD_OUTPUT  436    Output file not of proper type          

 ANGLE_TOO_BIG     501    celestial angle too large for projection 
 BAD_WCS_VAL       502    bad celestial coordinate or pixel value 
 WCS_ERROR         503    error in celestial coordinate calculation  
 BAD_WCS_PROJ      504    unsupported type of celestial projection 
 NO_WCS_KEY        505    celestial coordinate keywords not found
 APPROX_WCS_KEY    506    approximate wcs keyword values were returned
\end{verbatim}

\end{document}
